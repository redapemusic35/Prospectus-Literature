\PassOptionsToPackage{unicode=true}{hyperref} % options for packages loaded elsewhere
\PassOptionsToPackage{hyphens}{url}
%
\documentclass[]{article}
\usepackage{lmodern}
\usepackage{amssymb,amsmath}
\usepackage{ifxetex,ifluatex}
\usepackage{fixltx2e} % provides \textsubscript
\ifnum 0\ifxetex 1\fi\ifluatex 1\fi=0 % if pdftex
  \usepackage[T1]{fontenc}
  \usepackage[utf8]{inputenc}
  \usepackage{textcomp} % provides euro and other symbols
\else % if luatex or xelatex
  \usepackage{unicode-math}
  \defaultfontfeatures{Ligatures=TeX,Scale=MatchLowercase}
\fi
% use upquote if available, for straight quotes in verbatim environments
\IfFileExists{upquote.sty}{\usepackage{upquote}}{}
% use microtype if available
\IfFileExists{microtype.sty}{%
\usepackage[]{microtype}
\UseMicrotypeSet[protrusion]{basicmath} % disable protrusion for tt fonts
}{}
\IfFileExists{parskip.sty}{%
\usepackage{parskip}
}{% else
\setlength{\parindent}{0pt}
\setlength{\parskip}{6pt plus 2pt minus 1pt}
}
\usepackage{hyperref}
\hypersetup{
            pdfborder={0 0 0},
            breaklinks=true}
\urlstyle{same}  % don't use monospace font for urls
\usepackage[margin=1in]{geometry}
\usepackage{graphicx,grffile}
\makeatletter
\def\maxwidth{\ifdim\Gin@nat@width>\linewidth\linewidth\else\Gin@nat@width\fi}
\def\maxheight{\ifdim\Gin@nat@height>\textheight\textheight\else\Gin@nat@height\fi}
\makeatother
% Scale images if necessary, so that they will not overflow the page
% margins by default, and it is still possible to overwrite the defaults
% using explicit options in \includegraphics[width, height, ...]{}
\setkeys{Gin}{width=\maxwidth,height=\maxheight,keepaspectratio}
\setlength{\emergencystretch}{3em}  % prevent overfull lines
\providecommand{\tightlist}{%
  \setlength{\itemsep}{0pt}\setlength{\parskip}{0pt}}
\setcounter{secnumdepth}{0}
% Redefines (sub)paragraphs to behave more like sections
\ifx\paragraph\undefined\else
\let\oldparagraph\paragraph
\renewcommand{\paragraph}[1]{\oldparagraph{#1}\mbox{}}
\fi
\ifx\subparagraph\undefined\else
\let\oldsubparagraph\subparagraph
\renewcommand{\subparagraph}[1]{\oldsubparagraph{#1}\mbox{}}
\fi

% set default figure placement to htbp
\makeatletter
\def\fps@figure{htbp}
\makeatother


\author{}
\date{\vspace{-2.5em}}

\begin{document}

\hypertarget{header-wrapper}{}
\hypertarget{header}{}
\hypertarget{branding}{}
\leavevmode\hypertarget{site-logo}{}%

\leavevmode\hypertarget{site-title}{}%
Stanford Encyclopedia of Philosophy

\hypertarget{navigation}{}
\begin{navbar}

\begin{navbar-inner}

\begin{container}

 Menu

\begin{nav-collapse}

 Browse

Table of Contents

What's New

Random Entry

Chronological

Archives

 About

Editorial Information

About the SEP

Editorial Board

How to Cite the SEP

Special Characters

Advanced Tools

Contact

 Support SEP

Support the SEP

PDFs for SEP Friends

Make a Donation

SEPIA for Libraries

\end{nav-collapse}

\end{container}

\end{navbar-inner}

\end{navbar}

\hypertarget{search}{}
\begin{search-btn-wrapper}

\end{search-btn-wrapper}

\hypertarget{article-sidebar}{}
\begin{sticky}

\begin{navbar}

\begin{navbar-inner}

\begin{container}

 Entry Navigation

\hypertarget{article-nav}{}
\begin{nav-collapse}

Entry Contents

Bibliography

Academic Tools

Friends PDF Preview

Author and Citation Info

Back to Top

\end{nav-collapse}

\end{container}

\end{navbar-inner}

\end{navbar}

\end{sticky}

Reflective Equilibrium

\leavevmode\hypertarget{pubinfo}{}%
First published Mon Apr 28, 2003; substantive revision Fri Oct 14, 2016

\hypertarget{preamble}{}
Many of us, perhaps all of us, have examined our moral judgments about a
particular issue by looking for their coherence with our beliefs about
similar cases and our beliefs about a broader range of moral and factual
issues. In this everyday practice, we have sought ``reflective
equilibrium'' among these various beliefs as a way of clarifying for
ourselves just what we ought to do. In addition, we may also have been
persuading ourselves that our conclusions were justifiable and
ultimately acceptable to us by seeking coherence among them. Even though
it is part of our everyday practice, is this approach to deliberating
about what is right and finding justification for our views defensible?

Viewed most generally, a ``reflective equilibrium'' is the end-point of
a deliberative process in which we reflect on and revise our beliefs
about an area of inquiry, moral or non-moral. The inquiry might be as
specific as the moral question, ``What is the right thing to do in this
case?'' or the logical question, ``Is this the correct inference to
make?'' Alternatively, the inquiry might be much more general, asking
which theory or account of justice or right action we should accept, or
which principles of inductive reasoning we should use. We can also refer
to the process or method itself as the ``method of reflective
equilibrium.''

The method of reflective equilibrium can be carried out by individuals
acting separately or together. In the latter case, the method is
dialogical and agreement among participants may or may not be
accompanied by a search for coherence. We shall be focused on the method
when it seeks coherence among beliefs, for an avowal of agreement may
well not include a real coherence of beliefs.

In what follows, we first give an overview of the method of reflective
equilibrium and comment briefly on its history. We then discuss in more
detail the evolution of the method and its role in the work of John
Rawls. Against that background, we then remark on some of the
controversy surrounding the claim that coherence among our moral or our
logical beliefs in reflective equilibrium counts as a justification for
them, including the challenge to including moral intuitions or facts
about the world in such a justification. Finally, we discuss some
implications the method has for work in ethics.

1. The Method of Reflective Equilibrium

2. Recent History

2.1 Origins in justification of logic

2.2 Development in Ethics and Political Philosophy

3. Distinguishing Narrow from Wide Reflective Equilibrium

3.1 Narrow Reflective Equilibrium

3.2 Wide Reflective Equilibrium

4. Criticisms of Reflective Equilibrium

4.1 Challenges to Moral Intuitions

4.2 Rejection of Rawls's Constructivism and Wide Reflective Equilibrium

4.3 Epistemological Criticisms of Reflective Equilibrium

5. Applications and Implications of Reflective Equilibrium

Bibliography

Academic Tools

Other Internet Resources

Related Entries

\hypertarget{main-text}{}
\protect\hypertarget{MetRefEqu}{}{1. The Method of Reflective
Equilibrium}

The method of reflective equilibrium consists in working back and forth
among our considered judgments (some say our ``intuitions,'' though
Rawls (1971), the namer of the method, avoided the term ``intuitions''
in this context) about particular instances or cases, the principles or
rules that we believe govern them, and the theoretical considerations
that we believe bear on accepting these considered judgments,
principles, or rules, revising any of these elements wherever necessary
in order to achieve an acceptable coherence among them. The method
succeeds and we achieve reflective equilibrium when we arrive at an
acceptable coherence among these beliefs. An acceptable coherence
requires that our beliefs not only be consistent with each other (a weak
requirement), but that some of these beliefs provide support or provide
a best explanation for others. Moreover, in the process we may not only
modify prior beliefs but add new beliefs as well. There need be no
assurance the reflective equilibrium is stable---we may modify it as new
elements arise in our thinking (Schroeter 2004). In practical contexts,
this deliberation may help us come to a conclusion about what we ought
to do when we had not at all been sure earlier. (Scanlon 2002). We
arrive at an optimal equilibrium when the component judgments,
principles, and theories are ones we are un-inclined to revise any
further because together they have the highest degree of acceptability
or credibility for us. An alternative account retains the importance of
revisability and emphasizes the positive role of examining our moral
intuitions, but rejects the appeal to coherentism in favor a treating
our intuitive moral judgments as the right sort to count as
foundational, even if they are still defeasible (McMahan 2000, Nichols
2012).

The method of reflective equilibrium has been advocated as a coherence
account of justification (as contrasted with an account of truth) in
several areas of inquiry, including inductive and deductive logic as
well as both theoretical and applied philosophy. The key idea underlying
this view of justification is that we ``test'' various parts of our
system of beliefs against the other beliefs we hold, looking for ways in
which some of these beliefs support others, seeking coherence among the
widest set of beliefs, and revising and refining them at all levels when
challenges to some arise from others. For example, a moral principle or
moral judgment about a particular case (or, alternatively, a rule of
inductive or deductive inference or a particular inference) would be
justified if it cohered with the rest of our beliefs about right action
(or correct inferences) on due reflection and after appropriate
revisions throughout our system of beliefs. By extension of this
account, a person who holds a principle or judgment in reflective
equilibrium with other relevant beliefs can be said to be justified in
believing that principle or judgment.

Because we are expected to revise our beliefs at all levels as we work
back and forth among them and subject them to various criticisms, this
coherence view contrasts sharply with a variety of many foundationalist
approaches to justification. In ethics, some foundationalist approaches
take some subset of our moral beliefs as fixed or unrevisable. Other
foundationalists at least claim that some subset of our moral beliefs
are immediately or directly justified (perhaps even ``self-evident'') or
warranted (leaving aside the issue of revisability) and serve as the
basis on which all other beliefs are justified. Still others take some
subset of our beliefs as at least justifiable independently of any other
moral beliefs, even if they are justifiable in light of either necessary
or contingent views of the person or human nature or through appeals to
the logic of moral discourse (Timmons 1987). Reflective equilibrium is
unlikely to single out any such group of privileged or directly
justified beliefs, distinguishing itself from all these forms of
foundationalism. Rawls (1974) thought it highly unlikely, but not
impossible, that moral principles could be formulated so compellingly
(be ``self-evident'') that we would favor them and their consequences
over all our previously held considered moral judgments; he thus leaves
foundationalism a slim possibility, a point seized on by some who are
willing to ignore Rawls's probability claim that such foundationalism is
not probable at all. If we take this probability judgment seriously,
then the warrant for a belief lies (with high probability) in its
coherence with other beliefs and not in its resting on beliefs for which
foundationalist claims are made.

Because it is not foundationalist in these ways, reflective equilibrium
also avoids some other problematic distinctions or claims that are part
of an effort to show how some beliefs can be directly justified or
warranted. For example, some foundationalists view particular moral
judgments as fixed; others might think it is our moral principles or
some deeper theoretical beliefs from which such principles might be
derived that are fixed and unrevisable. Some proponents of both
approaches have even claimed that a moral sense or faculty reveals these
directly justified beliefs to us. For others, we can discover the
foundational beliefs in some deep moral belief structure that is
revealed to us through a careful examination of moral judgements and
that is arguably a priori. A considerable part of contemporary work in
substantive ethics treats appeals to moral intuitions or considered
judgments in this way (Thompson 1976, McMahan 2000, G.A. Cohen 2007).
Some who work in this fashion cite Rawls (1974) when they embrace
reflective equilibrium. We return to the topic of intuitionism again
later in this entry.

In contrast, advocates of reflective equilibrium need tell no
controversial stories about credentials for a special subset of beliefs
that is directly justifiable. Where such advocates credit some initial
beliefs with strong initial acceptance, as Rawls (Rawls 1971) does when
he refers to some beliefs as initial ``fixed points,'' the beliefs
remain revisable; in any case, they are not taken to be points we
believe independently of other moral and non-moral beliefs we hold, and
so they have no special foundations (Harman 2003). This revisability of
the initial beliefs and their dependence on other beliefs when reasons
for them are requested means that no such special epistemological story
has to be told about them.

As we shall see, however, an important point of controversy, especially
in ethics, is not that reflective equilibrium allows for the revision of
all moral judgments but, rather, that it involves giving some initial
justificatory weight to them at all.

\protect\hypertarget{RecHis}{}{2. Recent History}

\protect\hypertarget{OriJusLog}{}{2.1 Origins in justification of logic}

This approach to the justification of rules of inductive logic---without
the label ``reflective equilibrium''---was proposed by Nelson Goodman in
his classic Fact, Fiction, and Forecast (Goodman 1955). Goodman's idea
was that we justify rules of inference in inductive or deductive logic
by bringing them into reflective equilibrium with what we judge to be
acceptable inferences in a broad range of particular cases. No rule of
inference would be acceptable as a logical principle if it was not
compatible with what we take to be acceptable instances of inferential
reasoning. In this sense, our beliefs about acceptable rules of
inference are constrained by the ``evidence'' provided by what we
believe to be good or correct examples or instances of inferential
reasoning. At the same time, we should correct or revise our views about
particular inferences we initially might think are acceptable if we come
to see them as incompatible with rules that we generally accept and
refuse to reject because they, in turn, best account for a broad range
of other acceptable inferences.

Some have criticized this account as giving too much weight to our
actual inductive practices (Stich 1990, Kelly and McGrath 2010).
Obviously, not all elements of the everyday reasoning practices of all
individuals are justifiable. For example, many of us, to our chagrin,
have had to confess committing the error of the gamblers fallacy in our
own betting on games or on the events of life. Quite generally,
psychological studies reveal widespread errors in reasoning in a broad
range of contexts. More recently, others have suggested that reflective
equilibrium is problematic as a form of justification of inductive
reasoning because it is fragile as a method, allowing some features of
our beliefs to trigger significant changes in the equilibrium they
reach, and it provides inadequate assurance about its reliability, as a
way of telling us what beliefs to replace with other beliefs (Harman and
Kulkarni 2006).

Though Goodman thinks justification of our reasoning practices depends
on what inferences we accept when we reason inductively and deductively,
he is not simply seeking to systematize whatever inferences we happen to
find people sometimes---unreflectively---make. Instead, he insists that
practice can and should be corrected as we work back and forth from
tentative principles to practice, revising where appropriate, presumably
eliminating the sorts of inconsistencies that some psychological
studies, and our everyday experiences, reveal. But, some critics ask
(Siegel 1992), if practice should be revised, why should we view ``fit''
with practice as at all justificatory?

A more generous reading of Goodman's proposal would widen the reflective
equilibrium he proposes to include some of the beliefs about standards
for acceptable inference that logicians develop (though some think that
wide reflective equilibrium does not overcome the reliability problem
noted above). Such standards are themselves not independent of all
inferential practice. They have been developed to reflect views about
what counts as good practice in light of the kinds of inferences that
people abandon when they are made aware of their inconsistency with
other inferences they will not give up. It is against this wider set of
beliefs, including the articulation of such standards, that we can
identify some inferences as performance errors or otherwise deviant
patterns and correct those practices. Some changes in belief in
dialogical contexts may well be best explained by noting that some
individuals grasp the insights of others. Individuals working with the
method of reflective equilibrium may thus see the point of criticisms of
previous views they had accepted. This is where some dialogical and
individual uses of the method coincide. (We turn to the distinction
between wide and narrow reflective equilibrium shortly.)

\protect\hypertarget{DevEthPol}{}{2.2 Development in Ethics and
Political Philosophy}

Despite the fact that the origins of reflective equilibrium (minus the
name) lie in mid-twentieth century discussions about the justification
of inductive logic, its principal development through the rest of the
century lies primarily in ethics and political philosophy. Specifically,
the method was given prominence (and the name by which it is known) by
John Rawls's description and use of it in A Theory of Justice (Rawls
1971). (Rawls 1951 had much earlier articulated a slightly different
version of the view.) Although accounts of the justification of
empirical knowledge have been developed that share with reflective
equilibrium its coherentist approach, they generally do not make
explicit use of the terminology of reflective equilibrium, and we shall
not discuss them here. Instead, we concentrate on the use of reflective
equilibrium in ethics and political philosophy, where it has been
deployed and criticized.

Rawls (Rawls 1971) argues that the goal of a theory of justice is to
establish the terms of fair cooperation that should govern free and
equal moral agents. On this view, the appropriate perspective from which
to choose among competing conceptions or principles of justice is a
hypothetical social contract or choice situation in which contractors
are constrained in their knowledge, motivations, and tasks in specific
ways. Because this choice situation is fair to all participants, Rawls
calls the conception of justice that emerges from this choice ``justice
as fairness.'' Under these constraints, he argues, rational contractors
would choose principles guaranteeing equal basic liberties and equality
of opportunity, and a principle that permitted inequalities only if they
made the people who are worst off as well off as possible.

Instead of simply accepting whatever principles contractors would choose
under these constraints on choice, however, Rawls imposed a further
condition of adequacy on them. The chosen principles must also match our
considered judgments about justice in reflective equilibrium. If they do
not, then we are to revise the constraints on choice in the contract
situation until we arrive at a contract that yields principles that are
in reflective equilibrium with our considered judgments about justice.
This restriction constitutes a further assurance that the outcomes of
the deliberation about fair terms of cooperation in the choice situation
(the Original Position) actually count as focusing on justice rather
than on some other domain. In effect, the device of the contract must
itself be in reflective equilibrium with the rest of our beliefs about
justice. The contract helps us determine what principles we should
choose from among competing views, but the justification for using it
and designing it so that it serves that purpose must itself derive from
the reflective equilibrium that it helps us achieve.

The method of reflective equilibrium thus plays a role in both the
construction and justification of Rawls's theory of justice (Daniels
1996; Scanlon 2002). Its role in construction is an example of its use
as a form of deliberation. Critics of Rawls's theory and his method of
reflective equilibrium, especially utilitarians, challenge the
prominence the method gives to moral judgments or intuitions. Building a
theory---constructing it---out of such initial judgments is building it
on easily discredited bases, for are not many of our beliefs just the
result of historical accident and bias, even superstition? Indeed, some
argue that there is no justificatory force to the contract itself if we
``rig'' the contract so that it yields principles that match our
intuitions (Hare 1973). Despite these and other criticisms, defenders of
the method have elaborated it and extended its use in broad areas of
ethics. Later in this article we shall consider some of the criticisms
of this method and some of its extensions in more detail. Before doing
so, however, it is necessary to describe it more fully.

\protect\hypertarget{DisNarWidRefEqu}{}{3. Distinguishing Narrow from
Wide Reflective Equilibrium}

\protect\hypertarget{NarRefEqu}{}{3.1 Narrow Reflective Equilibrium}

A reflective equilibrium may be narrow or wide (Rawls 1974). All of us
are familiar with a process in moral deliberation in which we work back
and forth between a judgment we are inclined to make about right action
in a particular case and the reasons or principles we offer for that
judgment. Often we consider variations on the particular case,
``testing'' the principle against them, and then refining and specifying
it to accommodate our judgments about these variations. We might also
revise what we say about certain cases if our initial views do not fit
with the principles we grow inclined to accept.

Such a revision may constitute a moral surprise or discovery (Daniels
1996). Suppose, for example, that we are considering whether we should
ignore age in the distribution of medical treatments. Many initially
believe that age is a ``morally irrelevant trait,'' just like race, and
they would insist that rationing medical services by age is just as
unacceptable as rationing by race. On considering a variety of cases,
however, it might become apparent that we all age, although we do not
change race. This difference means that the different treatment of
people at different ages, if systematically applied over the lifespan,
does not create inequalities between persons, as it would in the case of
race. We might be led by this realization to think that age rationing
might be acceptable under some conditions when race rationing would
never be, and this would be a moral surprise for many who changed their
view.

To the extent that we focus solely on particular cases and a group of
principles that apply to them, and to the extent that we are not
subjecting the views we encounter to extensive criticism from
alternative moral perspectives, we are seeking only narrow reflective
equilibrium. Presumably, the principles we arrive at in narrow
equilibrium best ``account for'' the cases examined. Others, however,
may arrive at different narrow reflective equilibria, containing
different principles and judgments about justice. Indeed, one such
narrow equilibrium might be characterized as typically utilitarian,
while another is, we may suppose, Kantian or perhaps Libertarian. As a
result, we still face an important question about justification
unanswered by the method of narrow reflective equilibrium: which set of
beliefs about justice should we accept?

Because narrow reflective equilibrium does not answer this question, it
may seem to be a descriptive method appropriate to moral anthropology,
not a normative account of justification in ethics. In fact, Rawls
(Rawls 1971) at one point suggested that arriving at the principles that
match our moral judgments in reflective equilibrium might reveal our
``moral grammar'' in a way that is analogous to uncovering the grammar
that underlies our syntactic ability as native speakers of a language to
make judgments about grammatical form. In support of the analogy, some
contemporary theorists who systematically examine our moral intuitions,
often through hypothetical as opposed to real cases, believe they are
uncovering an underlying moral structure of principles (see Kamm 1993),
perhaps one that is a priori.

Uncovering a syntax, however, is a descriptive and not a justificatory
task. Once we can identify the grammar or rules that best account for a
person's syntactic competency, we do not ask the question, Should that
person have this grammar? We are satisfied to have captured the grammar
underlying a person's idiolect. In ethics and political philosophy, in
contrast, we must answer that justificatory question, especially since
there is often disagreement among people about what is right,
disagreement that is not resolved simply by pursuing narrow reflective
equilibrium.

In A Theory of Justice (1971, revised edition 1999), Rawls does not use
the terminology of narrow and wide reflective equilibrium, an omission
he remarks about with regret in Justice as Fairness: A Restatement
(2001, p.~31). Still, he comments that seeking a reflective equilibrium
that merely irons out minor incoherence in a person's system of beliefs
is not really the use of the method that is of true philosophical
interest in ethics, just as we saw it might not be in the case of
justifying logical inferences. Rather, he says, to be of interest to
moral philosophy, a reflective equilibrium should seek what results from
challenging existing beliefs by arguments and implications that derive
from the panoply of developed positions in moral and political
philosophy (Rawls A Theory of Justice 2nd Edition, 1999, p.~43). Such a
reflective equilibrium would be the response to considerable critical
pressures on the original beliefs. This effort would have the character
of searching deliberation about what is right. It is this much broader
form of challenge that Rawls labels the method of wide reflective
equilibrium.

\protect\hypertarget{WidRefEqu}{}{3.2 Wide Reflective Equilibrium}

3.2.1 A Theory of Justice

Rawls's proposal is that we can determine what principles of justice we
ought to adopt, on full reflection, and be persuaded that our choices
are justifiable to ourselves and others, only if we broaden the circle
of beliefs that must cohere. Indeed, we should be willing to test our
beliefs against developed moral theories of various types, obviously not
all such views (as Arras 2007, Strong (2010) comment, Kelly and McGrath
2010), or we would never arrive at a conclusion, but at least against
some leading alternatives. (We often do so in discussion or deliberation
with others. How conclusive an argument or deliberation is may depend on
what alternative views are considered or on who is included in the
deliberation.) In effect, Rawls's (Rawls 1971) pair-wise comparison of
justice as fairness with utilitarianism is just such an exercise at
seeking wide reflective equilibrium. In a wide reflective equilibrium,
for example, we broaden the field of relevant moral and non-moral
beliefs (including general social theory) to include an account of the
conditions under which it would be fair for reasonable people to choose
among competing principles, as well as evidence that the resulting
principles constitute a feasible or stable conception of justice, that
is, that people could sustain their commitment to such principles.
Rawls's argument is that justice as fairness, rather than
utilitarianism, is what emerges in wide reflective equilibrium.

For example, the constraints on choice in Rawls's contract situation,
such as the veil of ignorance that keeps us from knowing facts about
ourselves and our specific preferences, require justification. Rawls
needs to show that these constraints are ``fair'' to all contractors
(thus ``justice as fairness'' is his label for this procedural account
of how we find out what is just.) To provide such justification, Rawls
appeals to beliefs about the fundamental ``moral powers'' of ``free and
equal'' agents (they can form and revise their conceptions of what is
good and they have a sense of justice). He also appeals to the ideal of
a well-ordered society in which principles of justice play a particular
role as public principles for reconciling disputes. Rawls must even
provide us with an account of ``primary social goods,'' necessary if
agents who do not know their own actual preferences are to decide what
principles would be better for them to choose.

The device of the contract is thus in reflective equilibrium with
certain background theories that themselves contain moral beliefs. These
are theories, or crucial beliefs, about the nature of persons, the role
of morality or justice in society, and beliefs about procedural justice.
The contract is not simply based on uncontroversial assumptions about
human rationality, although it describes a rational choice problem
within the constraints it imposes; nor is it based on formal
considerations about practical reasoning or the logic or semantics of
moral discourse. If Rawls were trying to justify the structure of the
contract by appeal to theories that themselves were completely
non-moral, then he would be offering the kind of independent
justification for the principles that would characterize them as
foundational (Daniels 1996, Timmons 1987), so the claim that the
background theories are themselves moral is part of the rationale for
concluding that Rawls is clearly rejecting foundationalism. (Some
philosophers, as noted earlier, argue that reflective equilibrium is
compatible with foundationalism, citing Rawls in this regard though they
ignore his probability judgment that such a compatibility exists in
principle and is highly unlikely to be met.) Without acceptance of that
wider circle of moral beliefs, Rawls's construction lacks support, and
if that construction were modified, some of his arguments against
utilitarianism would be weakened.

Our beliefs about justice are justified (and, by extension, we are
justified in holding them) if they cohere in such a wide reflective
equilibrium. Obviously, the method of wide reflective equilibrium is
here only illustrated by appeal to the detail of Rawls's use of it. If
we abstract from that detail, we see that there is a complex structure
and interaction of beliefs, at many levels of generality, that bear on
the construction of an account of justice. We shall later return to this
point when we talk about some of the criticisms of method and about
implications of this method for work in ethics.

3.2.2 Political Liberalism and `justice as political'

In A Theory of Justice, Rawls seemed to think that all people might
converge (but see Kelly and McGrath 2010) on a common or shared wide
reflective equilibrium that included ``justice as fairness,'' the
conception of justice for which he argues. Wide reflective equilibrium
thus played a role in the construction of the theory, helping
``us''---any and all of us---to articulate its key features in ways that
led to principles that matched ``our'' considered judgments. At the same
time, wide reflective equilibrium constituted an account of
justification. Shared agreement on that wide equilibrium would produce a
well-ordered society governed by principles guaranteeing equal basic
liberties, fair equality of opportunity, and the requirement that
inequalities be arranged to make those who are worst off as well off as
possible.

In his later work, Political Liberalism (Rawls 1993), Rawls abandons the
suggestion that all people might converge in the same, shared wide
reflective equilibrium that contains his conception of justice. This
important change comes about because of what Rawls calls ``the burdens
of judgment.'' Complexity, uncertainty, and variation in experience lead
human reason, when exercised under conditions of freedom, of the sort
protected by the principles of justice as fairness, to an unavoidable
pluralism of comprehensive moral and philosophical views. (The need to
find agreement in the form of an overlapping consensus of views arises
domestically under conditions of liberty and the burdens of judgment and
is evidenced further by the different perspectives that we might get
from a global view of issues.) This unavoidable fact of reasonable
pluralism makes one key feature of justice as fairness untenable, namely
the account Rawls gave of the stability of his preferred conception of
justice. A further feature of Rawls's late work is his clear eschewal of
any account of truth, so the fact that wide equilibrium is offered as an
account of justification, not truth, makes it compatible with the
refusal to discuss moral truth (see Little 1984).

Rawls had imposed three basic conditions on the principles of justice.
First, they must be chosen over alternatives under conditions fair to
all contractors. Second, what contractors choose must match ``our''
considered moral judgments and other beliefs in (wide) reflective
equilibrium. But third, the principles must comprise a feasible or
stable conception of justice. The test for stability is to ask if people
raised under this view would conform to it over time with less strain of
commitment than other conceptions would face. In effect, passing the
test shows it is worth adopting this view because it will not prove so
fragile that it is not worth the effort to institutionalize.

The problem facing Rawls because of reasonable pluralism is that his
earlier efforts to show justice as fairness was stable depended on
special views about the importance of autonomy that would be held only
by some people and could not be assumed to be shared. His argument for
stability had turned on appealing to the good of autonomy in ways that
might not be acceptable to people holding certain comprehensive views.
Stability was not demonstrated.

To address this problem, Rawls recasts justice as fairness as a
``freestanding'' political conception of justice on which people with
different comprehensive views may agree in an ``overlapping consensus.''
The public justification of such a political conception involves no
appeal to the philosophical or religious views that appear in the
comprehensive doctrines that form this overlapping consensus. Instead,
we might think of this process of working back and forth among the key
shared ideas in the public, democratic culture and the articulated
features of the political conception of justice as a political
reflective equilibrium (Daniels 1996). (We can imagine the process being
carried out by individuals, for example, as citizens thinking about an
issue of basic justice or judges about a matter of constitutional
essentials, or as a collective process, in public deliberation. As noted
in Table 1 below, people with different comprehensive world views may
draw the boundary between public and private slightly differently and
not all questions about basic justice will be settled in a way that
rules out ongoing disagreement. Thus disagreements about some aspects of
reproductive rights or the separation of church and state may remain
contested features within public reason.) The goal of this exercise of
what Rawls calls ``public reason'' is the articulation of such a
political reflective equilibrium.

Still, there is no convergence on a shared wide reflective equilibrium
that contains the political conception of justice. The political
reflective equilibrium is not a shared wide reflective equilibrium, for
it avoids appealing to the broad range of beliefs that would be included
in many wide reflective equilibria. Nevertheless, wide reflective
equilibrium still plays a critical role in justification even when
justice is political in this way.

For individuals to be fully justified in adopting the political
conception of justice, the conception articulated in the political
reflective equilibrium, they must incorporate it within a wide
reflective equilibrium that includes their own comprehensive moral or
religious doctrine. It will count as a reasonable view for them only if
the public conception of justice coheres with their other (religious and
philosophical) beliefs in wide reflective equilibrium. (Individuals
within a religion may find a religious basis for accommodating their
religion to concerns of public reason, though this outcome is unusual
unless the religious hierarchy in their religion has found such
accommodation as well.) Individuals, whatever groups and associations
they belong to, must be able to embark on the deliberative process
called wide reflective equilibrium that leads them to endorse, from
their own perspective, the content of the political reflective
equilibrium, including its injunctions about using only public reason in
thinking about certain basic features of justice or the constitution.

We can imagine use of wide reflective equilibrium if we modify slightly
Rawls image of an overlapping consensus in which a shared political
conception is a point of convergence among otherwise different
comprehensive world views. Convert each reasonable comprehensive world
view---for example, a Kantian, Millian, or a Protestant religious view
that endorses a notion of free faith---into a wide reflective
equilibrium that is reached by all who hold that view. These distinct
wide equilibria now share a module, the political reflective
equilibrium, though each wide equilibrium will justify the features of
that module in light of beliefs that may not be acceptable in other wide
equilibria.

To help see what has happened to the content of the wide reflective
equilibrium when we shift from the picture in A Theory of Justice to
that in Political Liberalism, consider Table 1 below.

Table 1: The contents of wide reflective equilibrium in Theory and
Liberalism (from Daniels 1996, p.~157).

Theory

Liberalism

Original Position; Principles of justice; free and equal agents;
well-ordered society; procedural justice; general social theory {[}n.b.
not freestanding module{]}

Original Position; Principles of justice; free and equal agents;
well-ordered society; procedural justice; general social theory {[}n.b.
freestanding module{]}

Philosophical arguments justifying key elements of justice as fairness,
such as argument about autonomy needed to show congruence between good
for individuals and what is just {[}n.b. share rationale{]}

Rationale for key elements of justice as fairness {[}n.b. rationale
specific to each wide reflective equilibrium{]}

~

Account of reasonableness and burdens of judgment

~

Rationale for boundary between public and nonpublic; specifics of
boundary.

Rest of moral and religious views

Rest of moral and religious views

Considered moral judgments, all levels

Considered moral judgments, all levels

The first row of the table indicates the content of justice as fairness,
including the background views that support the construction of the
original position. This ``module'' is contained in both versions of a
wide reflective equilibrium that justifies justice as fairness. In
Theory, as the second row indicates, shared philosophical arguments
justify key elements of the module, but in Liberalism, the rationale for
each of these elements must derive from the distinctive features of the
different comprehensive views. Thus, as Cohen suggests (1994, p.~1527),
a Kantian, a Millian, and a religious person who believed in free faith
might all support, but for quite different reasons, the idea that agents
were free in the sense of being capable of forming and revising their
conceptions of the good life.

As the third and fourth rows suggest, full justification of a political
conception of justice requires acceptance of an account of the burdens
of judgment that explain why pluralism is a fact of life. A
comprehensive view capable of reaching accommodation with that pluralism
and with the political reflective equilibrium it shares with other
reasonable views must also develop a rationale for the specific way in
which it draws a boundary between public and nonpublic spheres.
Different comprehensive views, however, might vary in just how they draw
that public/private boundary and there may be persistent controversy
about how permeable it is. Nevertheless, the freestanding public
conception of justice as fairness, for example, will be fully justified
for anyone in any of these different wide reflective equilibria.

It would be natural to object that if we require that full justification
for individuals includes appeal to the varied religious and
philosophical beliefs they hold, then overlapping consensus is indeed
harder to achieve. Remember, Rawls has abandoned a shared wide
reflective equilibrium for just such reasons. Overlapping consensus is
possible, however, because groups sharing comprehensive views modify the
content of their comprehensive views over time in order to cooperate
within shared democratic institutions. This process, on Rawls's view,
involves philosophical reflection, which often draws on the complex
resources of a tradition of thought in which disagreements had
flourished. It also crucially depends on the moderating influence of
living under democratic institutions that are governed by the shared
conception of justice. The effect of both institutions and reflection
about internal disagreements about doctrine is that reasonable
comprehensive world views can find room, from within their own
perspectives, for a wide reflective equilibrium that includes the
elements of public reason (the political reflective equilibrium) and a
willingness to engage in public methods of justification for them.

In sum, even though pluralism requires that we refrain from appealing to
comprehensive world views in certain areas of political deliberation
with others, wide reflective equilibrium remains at the center of Rawls
account of individual moral deliberation about justice. It survives as
the coherentist account of ``full justification'' he defends.

\protect\hypertarget{CriRefEqu}{}{4. Criticisms of Reflective
Equilibrium}

Key criticisms of the method of reflective equilibrium have challenged
the role it ascribes to moral intuitions, Rawls's incorporation of
empirical facts (e.g., about human nature, such as the role of
incentives in motivation), and its coherentism. We take these up in
turn.

\protect\hypertarget{ChaMorInt}{}{4.1 Challenges to Moral Intuitions}

Rawls does not talk about moral intuitions as the starting material for
the method of reflective equilibrium; instead he talks about considered
moral judgments. Some commentators think the whole issue of the relation
between the method of reflective equilibrium and moral intuitions is
resolved by noting that Rawls does not identify considered moral
judgments with an appeal to moral intuitions. But this ignores the
criticism that can be focused on considered moral judgments as well as
on moral intuitions. Central to the method of reflective equilibrium in
ethics and political philosophy is the claim that our considered moral
judgments about particular cases carry weight, if only initial weight,
in seeking justification. This claim is controversial. Some of the most
vigorous criticism of it has come from utilitarians, and it is
instructive to see why.

A traditional criticism of utilitarianism is that it leads us to moral
judgments about what is right that conflict with our ``ordinary'' moral
judgments. In response, some utilitarians accept the relevance of some
of these judgments and argue that utilitarianism is compatible with
them. Thus Mill argued for a utilitarian foundation for our beliefs
about the importance of individual liberty. Some utilitarians have even
argued that key features of our ``common sense morality'' approximates
utilitarian requirements and we have acquired these beliefs just because
they do, unconsciously, reflect what promotes utility; they reflect the
wisdom of a heritage.

An alternative utilitarian response to the claim that utilitarianism
conflicts with certain ordinary moral judgments is to dismiss these
judgments as pre-theoretical views---whether they are referred to as
``intuitions'' or ``considered judgments'' that probably result from
cultural indoctrination and thus reflect superstition, bias, and mere
historical accident. On this view, neither moral intuitions nor
judgments should have evidentiary credentials and should play no role in
moral theory construction or justification. Indeed, the prominent
20th-Century utilitarians Richard Brandt (Brandt 1979) and Richard Hare
(Hare 1973) argued against Rawls, simply making ``coherent'' a set of
beliefs that have no ``initial credibility'' cannot produce
justification, since coherent fictions are still only fictions. Indeed,
the conditional that Rawls describes under the process that we solicit
considered moral judgments, namely that people be calm and have adequate
information about the cases, do not by themselves do anything to assuage
the utilitarian worries. Brandt (Brandt 1990) reaffirms his early
criticism when he claims that considered judgments lack ``evidential
force'' regarding a moral order and therefore coherence in reflective
equilibrium has only a kind of persuasiveness that comes from coherence
among many elements being more convincing than the conviction that comes
from any of its parts.

This criticism has some---but not decisive--- force, since two standard
ways of supplying credentials for initial judgments are not available.
One traditional way to support the reliability of these judgments or
intuitions is to claim, as 18th century theorists did, that they are the
result of a special moral faculty that allows us to grasp particular
moral facts or universal principles. Modern proponents of reflective
equilibrium reject such mysterious faculties. Indeed, they claim moral
judgments are revisable, not foundational.

A second way to support the initial credibility of considered judgments
is to draw an analogy between them and observations in science or
everyday life. For example, what counts as observational evidence in
science depends on theory, and theory may give us reasons to reject some
observations as not constituting counter-evidence to a scientific law or
theory. In this way we might see an analogy between the revisability of
moral judgments and observations.

Developing this analogy, however, seems to require that we also tell
some story about why moral judgments are reliable ``observations'' about
what is right. The foundationalist view of moral intuitions held by
McMahan (2000) and perhaps other recent theorists who draw extensively
on moral intuitions in their work in substantive ethics also demands
some story about their reliability and so does not avoid this
difficulty, even if the view also allows for defeasibility of some of
the foundational intuitions. Perhaps we might need something like the
causal story that some theorists of knowledge offer to explain the
reliability of observations. Since no such story is forthcoming,
opponents argue, proponents of reflective equilibrium must reject the
requirement or give up the analogy. Proponents of reflective equilibrium
might reject the requirement by suggesting it is premature to ask for
such a story in ethics or by claiming that we can provide no analogous
causal story for credible judgments we make in other areas, including
mathematics or logic. So the utilitarian objection is not conclusive.

It might seem that the burden of argument has shifted to advocates of
reflective equilibrium to show why ``initial credibility'' should be
ascribed to moral judgments or intuitions. Defenders of reflective
equilibrium may nevertheless reject this burden, arguing that critics,
especially utilitarian critics, actually face the same problem. For
example, Richard Brandt argues that ``facts and logic'' alone, and not
moral intuitions, should play a role in moral theory construction and
justification. On Brandt's view, we should choose moral principles when
they are based on desires that have been subjected to maximal criticism
by facts and logic alone (he calls it ``cognitive psychotherapy''). We
should avoid any appeal to moral intuitions that might infect this
critique.

The desires that Brandt appeals to, however, are themselves shaped and
influenced by the very same social structures that utilitarians complain
have biased and corrupted our moral judgments. Nothing in the process of
critique by facts and logic alone can eliminate this source of bias. If
this utilitarian claim is right, it undercuts the suggestion that we can
step outside our beliefs to arrive at some more objective form of
justification. Instead, we may be better off recognizing that our
process of critique---in the method of wide reflective equilibrium---is
explicit about exposing the sources of bias and historical accident to
criticism and revision, rather than fooling ourselves into thinking that
desires are some sort of morally-uninfluenced layer of ``facts.''

\protect\hypertarget{RejRawConWidRefEqu}{}{4.2 Rejection of Rawls's
Constructivism and Wide Reflective Equilibrium}

Rawls claims that his view of justice is constructivist, meaning that he
appeals to some general claims about the nature of persons as well as
some empirical facts about human behavior or institutions as part of the
justification for the principles of justice (or the choice situation
that leads us to pick them). G.A. Cohen (2008) criticizes
constructivism, so understood, as not being capable of articulating the
content of what justice itself requires and instead being suited only
for selecting rules of regulation for society. The problem, he argues,
is that constructivism combines considerations of justice with other
considerations (both empirical and moral). As a result, it does not tell
us what justice itself requires. Specifically, he argues that if
considerations of justice plus other values and some empirical facts
suggest that we should adopt certain rules to regulate our institutions,
those rules cannot be principles of justice, since other considerations
than justice contribute to our thinking that we should adopt them. A
possible reply to Cohen is that his view about constructivism collapses
into his controversial metaethical claim that principles of justice
cannot rest on general facts about human behavior or anything else, if
the other values that we consider are appropriately focused on
determining what we think about justice. Then the other considerations
that continue to play a role are appeals to empirical facts.

To understand why the objection to constructivism collapses into Cohen's
metaethical view, if any values relevant to choices in the Original
Position are relevant to determining what justice requires, consider the
following. Cohen claims that, if we appeal to reflective equilibrium as
a justification of what justice requires, we need to strip away from it
any appeal to empirical facts and include only moral intuitions about
justice itself (Cohen 2007: p.~243, n. 19). Specifically, he is saying
that we might choose the right account of justice itself through appeal
to a narrow reflective equilibrium that avoids any empirical
considerations, and this is true even if the account of justice that
applies to our world does involve some relevant facts. This restriction
on reflective equilibrium implies that we would have to revise the
account of the role of wide reflective equilibrium that we earlier
attributed to Rawls, specifically that it, and not narrow reflective
equilibrium, plays a justificatory role in supporting justice as
fairness. For example, we would have to abandon the argument made
earlier that inadequate critical considerations play a role in narrow
reflective equilibrium as compared with wide reflective equilibrium.

Rawls insists that the principles that emerge from the Original Position
must fit with our considered judgments about justice in wide reflective
equilibrium. This constraint on the outcome of the Original Position is
aimed at showing that these principles are really principles of justice.
On Cohen's view this constraint could not show what Rawls intends--the
outcome for him of the choices made in the Original Position are rules
of regulation, not principles of justice. But if Cohen is wrong (as the
objection we are examining implies) about a broader set of values than
justice being in play in the choices made in the Original Position, then
his objection to constructivism collapses into his metaethical view, for
it is an objection to appealing to the facts that are part of the wide
reflective equilibrium to which Rawls does appeal. Similarly, Rawls's
overall account of justification makes the relative stability of a
conception of justice (given facts about human nature and behavior) part
of the justification for his principles of justice as fairness, but
Cohen concludes that this role is inappropriate.

If Cohen is right, then wide reflective equilibrium is not relevant to
choosing between Rawls's account of justice and some other theory (such
as luck egalitarianism), because we must restrict ourselves to a
different, much narrower, version of reflective equilibrium to justify
principles of justice. It would be odd to make such a substantive issue
depend on a metaethical claim that Cohen himself says is not a
substantive claim about which principles of justice are to be believed
or accepted.

\protect\hypertarget{EpiCriRefEqu}{}{4.3 Epistemological Criticisms of
Reflective Equilibrium}

Other criticisms of reflective equilibrium in ethics have focused on
points familiar from discussions in epistemology more generally. One
important complaint concerns the vagueness of the concept of coherence.
If we simply take logical consistency as the criterion for coherence, we
have much too weak a constraint. What account can we give of a stronger
notion?

More has to be said about how some parts of the system of beliefs
``support or explain'' others than is provided in the account above.
Critics of Rawls's requirement that the contract situation be adjusted,
if necessary, to yield principles that are in reflective equilibrium
with our judgments about justice complained that this was a ``rigged''
contract and that it did no justificatory work beyond reflective
equilibrium. If, however, the moral judgments that play a role in
discussions of fair process, the well-ordered society, and the moral
powers of agents are somewhat independent of the considered judgments
about justice, then we get the kind of independent support for the
principles that add justificatory force. Little work has actually been
done, however, to flesh out a stronger account of coherence in ethics as
opposed to epistemology more generally, where coherentism has been
defended and elaborated.

A second line of objection derives from anti-coherentist accounts of
justification rooted in the theory of knowledge. Some insist, for
example, that coherence accounts of justification cannot be divorced
from coherence accounts of truth. On the other hand, Rawls view fits
with the claim by some contemporary theorists of knowledge that a
coherence account of justification is distinguishable from a coherence
account of truth and defensible when so separated.

This separability of justification from claims about truth is important
to Rawls, even more so in his later work than in A Theory of Justice. In
his earlier work, Rawls never claimed that what emerged from his
contract and other justificatory conditions were ``truths'' of justice.
(Arras's (2007) conclusions emphasize the point that the method of
reflective equilibrium does not assure us of truth---but the point was
made explicit by Rawls himself decades earlier.) Rather, the method of
reflective equilibrium suggested ways in which convergence might be
achieved among those who began with disagreements about justice (of
course, as Arras (2007) and Kelly and McGrath (2010) emphasize, the
method of reflective equilibrium does not guarantee convergence either).
By drawing attention to the many features of a comprehensive theory on
which argument and evidence could be brought to bear, the method of wide
reflective equilibrium held out promise that more convergence might
result than if people had only considered judgments and principles to
agree and disagree about. Convergence did not imply that truth was
reached. A proponent of moral truth might still hope, however, that
convergence could be taken as evidence for truth, just as it commonly
claimed in non-normative areas of inquiry, contrary to the phenomenon
that believing may yield seeing.

Once Rawls took the ``burdens of judgment'' and reasonable pluralism
seriously, that is once he ``politicized'' justice in his later work, it
became much less plausible to talk about convergence on a shared wide
equilibrium that might contain moral truth. Comprehensive philosophical
views, including those about moral truth, were barred from playing a
role in seeking the political reflective equilibrium involved in
overlapping consensus. Even if we converged in an ``overlapping
consensus'' on a conception of justice, individuals and groups would be
able to claim they are fully justified in accepting that conception of
justice only if it cohered with the perspectives of their distinct wide
reflective equilibria. It became more important for Rawls to suggest how
claims about justice could be ``objective'' without presupposing moral
truth. These developments in Rawls's account thus appear to move him
farther away from those who would seek to give a realist account of
moral truth. Thus when Arras (2007) concludes that reflective
equilibrium cannot be a theory of truth, he does not disagree with
Rawls, as noted earlier. More challenging is his conclusion that
reflective equilibrium fails to provide a basis for intersubjective
agreement if we take seriously the importance of giving reflective
equilibrium a global scope, since starting points are so different.

Arras's point may be a version of a claim about human rationality. A
starting point for this line of argument is the complaint that people
who have different starting points as their initial set of beliefs, say
with different degrees of credence given to their beliefs, may arrive at
different reflective equilibria points---a specific version of which is
Arras's point about the global scope of claims of intersubjectivity.
This possibility undercuts the suggestion that critical pressures alone
deriving can produce convergence. A more general form of this worry is
that the model of reflection involved in reflective equilibrium, in
which we are to produce coherence among all our beliefs, overemphasizes
and idealizes human rationality. The suggestion is that we would be
better off presupposing a more minimal form of rationality. A specific
version of this complaint is that the method involves an information
burden that cannot be met. Yet another version of this criticism is that
we should allow for much less ``rationalist'' forms of modification of
our views, recognizing ``conversion'' experiences like those involved in
``paradigm shifts'' may affect any account of coherence. Another line of
argument emphasizes the information burden of the method---how can we be
sure all appropriate theoretical views have been addressed. The
suggestion (Arras 2007) is that wide reflective equilibrium is not
action guiding in that it does not tell us which views to save and which
to revise and so it does not lead to justification or truth.

A full defense of reflective equilibrium as a method would require a
more developed response to many of these lines of criticism than exists
in the literature.

\protect\hypertarget{AppImpRefEqu}{}{5. Applications and Implications of
Reflective Equilibrium}

Despite these criticisms, some philosophers have argued for a broader
understanding of the relevance of the method of reflective equilibrium
to practical ethics. In thinking about the course of right action in a
particular case, we often appeal to reasons and principles that are
notoriously general and lack the kind of specificity that make them
suitable to govern the case at hand without committing us to
implications we cannot accept in other contexts. This requires that we
refine or specify the reasons and principles if we are to provide
appropriate justifications for what we do and appropriate guidance for
related cases. Philosophers who have focused attention on the importance
of specification have drawn on the method of reflective equilibrium for
their insights into the problem.

In practical ethics, especially bioethics, there has been a vigorous
debate about methodology. Some argue that we must root all claims about
specific cases in specific ethical theories, and the hard work is
showing how these theories apply to specific cases. Others argue that we
may disagree about many aspects of general theory but still agree on
principles, and the hard work of practical ethics consists in fitting
sometimes conflicting principles to particular cases. Still others argue
that we must begin our philosophical work with detailed understanding of
the texture and specificity of a case, avoiding the temptation to
intrude general principles or theories into the analysis.

A grasp of the method of wide reflective equilibrium suggests a way
around this exclusionary nature of this debate. Wide reflective
equilibrium shows us the complex structure of justification in ethics
and political philosophy, revealing many connections among our component
beliefs. At the same time, there are many different types of ethical
analysis and normative inquiry.

This suggests a more eclectic view of the debate about method. Work in
ethics requires all levels of inquiry proposed by the disputants, not
always all at once or in each case, but at some time or other. Sometimes
it is true that we cannot resolve disputes about how to weigh conflicts
among principles unless we bring more theoretical considerations to bear
(these considerations need not involve comprehensive ethical theories).
Sometimes we can agree on relevant principles and agree to disagree on
other theoretical issues, still arriving at agreement about the
rightness of particular policies or actions and their justification in
light of relevant reasons and principles. Sometimes we must see what is
distinctive about particular cases and revise or refine our reasons and
principles before we can arrive at understanding of what to do. Finally,
and of the greatest significance, the method of wide reflective
equilibrium should make it clear that work in ethical theory cannot be
divorced from work in practical ethics. We must test and revise theory
in light of our considered judgments about moral practice. The
condescending attitude of many who work in ethical theory toward work in
practical ethics thus is incompatible with what wide reflective
equilibrium establishes about the relationship between these areas of
ethical inquiry.

The method of wide reflective equilibrium makes it plausible that all
such ``methods'' should be seen as appropriate to some tasks in ethics
and are but parts of a more encompassing method.

\hypertarget{bibliography}{}
\protect\hypertarget{Bib}{}{Bibliography}

Arras, J., 2007, `The Way We Reason Now: Reflective Equilibrium in
Bioethics', in The Oxford Handbook of Bioethics, B. Steinbock (ed.), New
York: Oxford University Press, pp.~46-71.

Brandt, R., 1979, A Theory of the Good and the Right, Oxford: Oxford
University Press.

------, 1990, `The Science of Man and Wide Reflective Equilibrium',
Ethics, 100: 259--278.

Cohen, G.A., 2008, Rescuing Justice and Equality, Cambridge: Harvard
University Press.

Cohen, J., 1994, `A More Democratic Liberalism', Michigan Law Review,
92(6): 1506--43.

Daniels, N., 1979, `Wide Reflective Equilibrium and Theory Acceptance in
Ethics', Journal of Philosophy, 76(5): 256--82; reprinted in Daniels,
N., 1996, Justice and Justification: Reflective Equilibrium in Theory
and Practice, Cambridge: Cambridge University Press, pp.~21--46.

------, 1996, Justice and Justification: Reflective Equilibrium in
Theory and Practice, New York: Cambridge University Press.

DePaul, M.R., 1993, Balance and Refinement: Beyond Coherence Methods of
Moral Inquiry, New York: Routledge.

Goodman, N., 1955, Fact, Fiction, and Forecast, Cambridge, MA: Harvard
University Press.

Hare, R.M., 1973, `Rawls' Theory of Justice', Philosophical Quarterly,
23: 144--55; 241--51.

Harman, G., 2003, `Three trends in moral and political philosophy',
Journal of Value Inquiry, 37: 415--25.

Harman, G., and Kulkarni, S., 2006, `The Problem of Induction',
Philosophy and Phenomenological Research, 52(3): 559--75.

Kamm, F., 1993, Morality and Mortality Vol. 1., Oxford: Oxford
University Press.

Kelly, T. and McGrath, S., 2010, `Is Reflective Equilibrium'
Philosophical Perspectives, 24(1): 325--359.

Little, D., 1984, `Reflective Equilibrium and Justification,' Southern
Journal of Philosophy, 22(3): 373--387.

McMahan, J., 2000, `Moral Intuition', in Blackwell Guide to Ethical
Theory, H. LaFollette (ed.), Oxford: Blackwell, chap. 5.

Nichols, P., 2012, `Wide reflective equilibrium as a method of
justification in bioethics,' Theoretical medicine and Bioethics, 33(5):
325-341.

Rawls, J., 1951, `Outline of a Decision Procedure for Ethics',
Philosophical Review, 60(2): 177--97, reprinted in Collected Papers,
1999, pp.~1--19.

------, 1971, A Theory of Justice, 2nd Edition 1999, Cambridge, MA:
Harvard University Press.

------, 1974, `The Independence of Moral Theory', Proceedings and
Addresses of the American Philosophical Association, 47: 5--22, in
Collected Papers, 1999, pp.~286--302.

------, 1993, 1996 Political Liberalism, New York: Columbia University
Press, Paperback Edition.

------, 1999, Collected Papers, Sam Freeman, (ed.), Cambridge MA:
Harvard University Press.

------, 2001, Justice as Fairness: A Restatement, Cambridge MA: Harvard
University Press.

Scanlon, T.M., 2002, `Rawls on Justification', in The Cambridge
Companion to Rawls, S. Freeman (ed.), Cambridge: Cambridge University
Press, pp.~139--167.

Schroeter, F., 2004, `Reflective Equilibrium and Anti-theory', Noûs,
38(1): 110-134.

Sencerz, S., 1986, `Moral Intuitions and Justification in Ethics',
Philosophical Studies, 50: 77--95.

Siegel, H., 1992, `Justification by Balance', Philosophy and
Phenomenological Research, 52(1): 27--46.

Stich, S., 1990, The Fragmentation of Reason, Cambridge, MA: MIT Press.

Strong, C., 2010, `Theoretical and Practical Problems with Wide
Reflective Equilibrium in Bioethics', Theoretical Medicine and
Bioethics, 31(20): 123--140.

Thompson, J.J., 1976, `Killing, Letting Die, and the Trolley Problem',
The Monist, 59: 204--217.

Timmons, M., 1987, `Foundationalism and the Structure of Ethical
Justification', Ethics, 97(3): 595--609.

van der Burg, W., and van Willigenburg, T., (eds.), 1998, Reflective
Equilibrium: Essays in Honour of Robert Heeger. Dordrecht, Kluwer
Academic Publishers.

\hypertarget{academic-tools}{}
\protect\hypertarget{Aca}{}{Academic Tools}

How to cite this entry.

Preview the PDF version of this entry at the Friends of the SEP Society.

Look up this entry topic at the Internet Philosophy Ontology Project
(InPhO).

Enhanced bibliography for this entry at PhilPapers, with links to its
database.

\hypertarget{other-internet-resources}{}
\protect\hypertarget{Oth}{}{Other Internet Resources}

{[}Please contact the author with suggestions.{]}

\hypertarget{related-entries}{}
\protect\hypertarget{Rel}{}{Related Entries}

justification, epistemic: coherentist theories of \textbar{} moral
epistemology \textbar{} original position \textbar{} pluralism
\textbar{} public reason \textbar{} Rawls, John \textbar{} social
contract: contemporary approaches to

\hypertarget{article-copyright}{}
Copyright © 2016 by Norman Daniels
\textless{}\href{mailto:ndaniels@hsph.harvard.edu}{\nolinkurl{ndaniels@hsph.harvard.edu}}\textgreater{}

\hypertarget{article-banner}{}
\begin{scroll-block}

\leavevmode\hypertarget{article-banner-content}{}%
Open access to the SEP is made possible by a world-wide funding
initiative. The Encyclopedia Now Needs Your Support Please Read How You
Can Help Keep the Encyclopedia Free

\end{scroll-block}

\hypertarget{footer}{}
\hypertarget{footer-menu}{}
\begin{menu-block}

 Browse

Table of Contents

What's New

Random Entry

Chronological

Archives

\end{menu-block}

\begin{menu-block}

 About

Editorial Information

About the SEP

Editorial Board

How to Cite the SEP

Special Characters

Advanced Tools

Contact

\end{menu-block}

\begin{menu-block}

 Support SEP

Support the SEP

PDFs for SEP Friends

Make a Donation

SEPIA for Libraries

\end{menu-block}

\hypertarget{mirrors}{}
\hypertarget{mirror-info}{}
 Mirror Sites

View this site from another server:

\begin{btn-group}

{} USA (Main Site) {} {CSLI, Stanford University}

Info about mirror sites

\end{btn-group}

\hypertarget{site-credits}{}
The Stanford Encyclopedia of Philosophy is copyright © 2020 by The
Metaphysics Research Lab, Center for the Study of Language and
Information (CSLI), Stanford University

Library of Congress Catalog Data: ISSN 1095-5054

\end{document}
